Vertebrates often need process sequential events. Sequential cognitive processing are driven by sequences of neural activity in various parts of the brain. The ubiquity of repeating sequential neural patterns prompts the speculation that the mechanisms of creation may be general across species. One test subject under frequent investigations is Zebra finch. Zebra finches are songbirds that produce sounds in a recurring fashion.  They can produce motifs that last up to 1 second, while single neuron bursts only sustain about 6 ms \cite{Hahnloser2002}. This leads the conclusion that the coding of sequential activity is sparse. 

The first model seeking to represent neural sequence generation is categorically referred as “synaptic chain” networks \cite{Kleinfeld1988}. The connectivity matrix in such networks is directional – neuronal groups are connected in a single file fashion. However, the network connectivity must be per-defined by hand, severely slashing its usability. 

One attempt to circumvent the drawback was made by Fiete and Hahnloser as described in the article \textit{Spike-Time-Dependent Plasticity and Heterosynaptic Competition Organize Networks to Produce Long Scale-Free Sequences of Neural Activity}. The authors claim that the combination of heterosynpatic competition within single neurons and spike-time-dependent plasticity (STDP) lead the network to sequential activity. 

STDP rules modify the synaptic connection so that the repeated timely causal activation of pair of neurons strengthens the connection, while the reverse is weakened \cite{Abbott2000}. Synchrony among firings arises naturally as a result. Because of this, a neuron that fires frequently will become a hub that modulates the firings in a network level. To form a long sequence, STDP alone is insufficient as suggested by the authors. Heterosynpatic competition was then introduced to supplement STDP. The underlying idea is  based on cellular biology and well documented - postsynaptic neurons conserve the total synaptic weight by depressing some input synapses when others undergo potentiation \cite{Royer2003}. The authors imposed a slightly different implementation onto their model – heterosynaptic long-term depression (hLTD) .


In this paper, we will re-examine Hahnloser \textit{et al.}'s claim that the combination of STDP and hLTD give rise to the generation of neural sequential firings. We will first investigate how strong of an effect hLTD imposes on the model by tuning relevant parameter. Then will we inspect the necessity of  STDP. 
