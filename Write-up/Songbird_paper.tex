\documentclass[12pt, draft]{article}
\usepackage[margin= 0.75in]{geometry}
\usepackage{amsfonts}
\usepackage{amsthm}
\usepackage{amsmath}
\usepackage{mathabx}
\usepackage{ulem}
\usepackage{enumerate}
\usepackage{float}
\usepackage{graphicx}
\usepackage{color}
\usepackage{subfigure}
\usepackage{commath}

\begin{document}

\newtheorem{prop}{Proposition}

\title{Think of a title later}
\author{Ansel Blumers and Ankan Ganguly}
\maketitle

\begin{abstract}
Abstract goes here.
\end{abstract}

\newpage

\tableofcontents

\newpage

\section{Introduction}

Write an introduction here.

\section{Methods}

In this paper, we used the simulated Integrate and Burst model introduced in \cite{Fiete}. We then implemented STDP learning and Hebbian learning.

\subsection{Integrate and Burst Model}

Our integrate and burst model was based off of the same equations as used in \cite{Fiete}. There are \(N\) neurons connected in an all-to-all environment. Each neuron \(i\) bursts when its membrane potential, \(V_i\), hits the threshold \(V_\theta\). We assume that every neuron bursts for \(T_{burst}\) time. While bursting, each neuron fires four times uniformly over the burst interval before resetting to \(V_{reset}\).

When a neuron is not bursting, its potential is governed by a typical conductance based leaky integrate and fire model with built-in inhibition:

\begin{equation}
\tau_V \od{V_i}{t} = -g^L(V - V^L) - g^E(V - V^E) - g^I(V - V^I)
\end{equation}

Where \(g^L\) is the leak conductance

\subsection{Learning}

Introduction to what we did.

\subsubsection{STDP Learning}

Talk about STDP learning in this implementation.

\subsubsection{Hebbian Learning}

Talk about Hebbian learning here.

\subsection{Parameter Choices}

Talk about how we chose parameters here.

\section{Results}

Introduce the big idea and what we got.

\subsection{Parameter Tuning}

\begin{itemize}
\item 4000 Hz doesn't work! (Burst Plot)

\item Mention annealing and our choice of \(r_{in}, \eta\) and \(\epsilon\). Name the two data sets we refer to for the remainder of the paper. (Scatter Error Function)

\item Setting \(w_{max}\). (Burst History)
\end{itemize}

\subsection{Convergence and Stability}

\begin{itemize}
\item Demonstrate the stability of our IB model by showing the firing rate plot and how it splits according to \(r_{in}\).

\item Plot Weight and \(WW^T\) for 4000 and 6000 Hz to show some level of convergence.

\item Plot error function over time from normal and from permutation matrix

\item Describe why the error function converges away from 0.
\end{itemize}

\subsection{Hebbian Learning versus STDP}

\begin{itemize}
\item Introduce the idea of the refutation. 
\item Give a theoretical description why the type of learning should be relatively unimportant.
\item Compare plots (\(WW^T\), Error vs Time, Burst History).
\end{itemize}

\section{Discussion}

Further improvements that could be made to our model and where this research could be taken.

\section{Summary}

Quick summary of our results and everything.



\bibliographystyle{plain}
\bibliography{Songbird_paper}

\end{document}